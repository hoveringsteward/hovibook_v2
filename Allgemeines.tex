% !TEX root = diplomarbeit.tex
\chapter{Einleitung}
\renewcommand{\kapitelautor}{Autor: Markus Kaiser}

%%%%%%%%%%%%%%%%%%%%%%%%%%%%%%%%%%%%%%%%%%%%%%%%%%%%%%%%%%%%%%%%%%%%%%%%%%%%%%%
\section{Projektidee}
Die Idee des Projektes, ist die Entwicklung eines innovativen Logistiksystems, das sich durch die Verwendung eines Multicopters auszeichnet.
Für den Entwicklungsprozess im Zuge der Diplomarbeit wurde der die Gastronomie als Einsatzbereich gewählt. Diese Wahl ermöglich zum Einen
ein konkreteres Fomulieren der Ziele und zum Anderen eine einheitliche Vision innerhalb des Teams.

Der innovative Aspekt des Projekts wird insofern abgedeckt, dass die Entwicklung des Systems so durchgeführt wird,
dass es auch in anderwärtigen Bereichen der Logistik, wie beispielsweise der Lagerverwaltung eingesetzt werden kann.

Das System setzt sich aus den drei Komponenten

\begin{itemize}
  \item Multicopter
  \item Firmware
  \item Digitale Speisekarte
\end{itemize}

zusammen.

(BILD: Systemblockbild)

Der Multicopter ist die zentrale Komponente des Projekts. Sie ist für den selbstständigen Transport von Objekten zuständig,
und muss somit flugfähig, und im Stande Unfälle zu verhindern sein.

Die Firmware regelt alle logischen Prozesse des Systems. Sowohl die Steuerung des Multicopters, als auch der
Austausch von Daten fällt unter diese Kategorie.

Die digitale Speisekarte ist indirekt die externe Steuereinheit des Multicopters, und zugleich die Komponente,
die das System für den Einsatz in der Gastronomie auszeichnet. Sie automatisiert Vorgänge eines gastronomischen
Betriebs und wandelt Nutzereingaben in Befehle für den Muticopter um.

%%%%%%%%%%%%%%%%%%%%%%%%%%%%%%%%%%%%%%%%%%%%%%%%%%%%%%%%%%%%%%%%%%%%%%%%%%%%%%%
\section{Ausgangssituation}
  Im den folgenden Absätzen wird beschrieben, wie die Idee von Hovering Steward, dem autonom fliegenden Kellner
  entstanden ist, und womit sich die ersten Recherchen zu Begin des Projektes beschäftigt, beziehungsweise
  welche Ergebnisse diese herausgebracht haben.

  \subsection{Ideenfindung}
  Ihren Anfang fand die Idee im Projektmanagement Unterricht. Die Teams mussten ein fiktives Projekt erfinden, um auf Basis dieses
  Projektmanagementpläne zu erstellen. Es entstand "Fluorescent Bakery", eine Bäckerei, die fluoreszierende Cupcakes verkauft.
  Der technische Teil ergab sich später, bei dem Gedanken diese Idee für die Diplomarbeit weiterzuverwenden. Der Ansatz hierfür war,
  einen nicht menschlichen Kellner zu erschaffen, der mithilife künstlicher Intelligenz die Aufgaben einer Bedienung in einem Restaurant übernimmt.
  Einen Kellner auf Rädern erschien zu simpel, daher wurde die Entscheidung getroffen, die 3. Dimension mit einzubeziehen und ihn fliegen zu lassen.
  So entstand "Hovering Steward - der autonom fliegende Kellner", ein dreidimensionales Tracking-System, welches eine Drohne in einem Raum die Aufgaben eines Kellners durchführen lässt.
  Das Projekt erhielt später noch eine weitere Komponente, nämlich eine digitale Speisekarte, um das System in einem Restaurant vollständig zu automatisieren.

  \subsection{Stand der Technik}
  \subsection*{Themenrestaurants}
  \begin{itemize}
    \item \textbf{Disaster Café}
    Das Themenrestaurant Disaster Café in Spanien bietet Gästen die Erfahrung, ihr Essen bei einem Erdbeben
    der Stärke 7.8 zu sich zu nehmen.

    \item \textbf{Das stille Örtchen - Modern Toilet}
    Eingerichtet wie eine Toilette, können Gäste des Modern Toilet in Japan ihre Speisen in einem gewohnten Umfeld genießen.

    \item \textbf{Affen-Kellner}
    Die kleine Taverne in Japan besitzt zwei kleine Affen, die Tätigkeiten wie das Servieren von Getränken oder Handtüchern
    für den Inhaber übernehmen. Sie verstehen sogar, welche Getränke ein Gast bestellt.

    \item \textbf{The Royal Dragon}
    Neben Rollschuh fahrenden Kellnern verfügt das weltweit größte Restaurant für Meeresfrüchte eine
    Art Seilbahn, mithilfe welcher ein Kellner "fliegend" das Essen serviert.

    \item \textbf{Dinner in the Sky}
    Bei Dinner in the Sky werden dee Gäste samt Tisch und kleiner Küche von einem Kran 50 Meter
    in die Luft gezogen, wo dann gespeist wird.

    \item \textbf{Dinner in the Dark}
    In diesem Themenrestaurant verbringen die Gäste ihren Aufenthalt im Dunkeln. Einzig und allein die
    Kellner können mittels Nachtsichtbrille etwas sehen.
  \end{itemize}

  \subsection*{Multicopter in der Gastronomie}
  \begin{itemize}
      \item \textbf{Infinium-Serve}
      Das Unternehmen Infinium-Robotics hat ein System entwickelt, welches Gästen eines Restaurants
      das Essen mittels Hexacopter serviert.

      \item \textbf{iTray}
      iTray, ein Projekt aus London, nutzt kleine Drohnen um Sushi an die Tische der Gäste zu bringen.
      Die Drohnen werden über ein Remote-Wi-Fi-System gesteuert.

  \end{itemize}

%%%%%%%%%%%%%%%%%%%%%%%%%%%%%%%%%%%%%%%%%%%%%%%%%%%%%%%%%%%%%%%%%%%%%%%%%%%%%%%
\section{Team und Aufgabenverteilung}
  \subsection*{Markus Kaiser}
  \textbf{Projektleitung und Marketing}

  Markus Kaiser leitete das Team Hovering Steward und war somit für das Projektmanagement und
  die organisatorischen Aspekte verantwortlich. Neben dieser Hauptrolle zählte außerdem das Marketing
  zu seinen Aufgabenbereichen, was speziell die Entwicklung des Blogs anbelangte. Durch seine Kenntnisse
  in der Webentwicklung stellte er mit dem Blog eine wichtige Schnittstelle zur Außenwelt her.

  \subsection*{Lucas Ullrich}
  \textbf{Sensorik \& Firmware}

  Lucas Ullrich war neben seiner Position als Projektleiter Stellvertreter für die Sensorik an unserer Drohne und
  für die Programmierung der Firmware zuständig. Er unterstützte unseren Projektleiter bei terminlichen Angelegenheiten
  und konnte durch seine Kenntnisse mit der Programmiersprache C eine solide Basis für die Firmware des Microcontrollers schaffen.
  Außerdem verfügte er über das notwendige Wissen und Engagement in der Elektrotechnik, um die Schaltpläne

  \subsection*{Christina Bornberg}
  \textbf{Firmware}

  Christina Bornberg fungierte als Firmware Entwicklerin des Teams. Durch ihre Interesse an der Programmierung von Drohnen
  und bereits gesammelten Erfahrung mit der Programmierung von C konnte sie gemeinsam mit Lucas Ullrich die Logik hinter
  dem automatisierten Flug der Drohne realisieren.

  \subsection*{Katharina Joksch}
  \textbf{Webentwicklung}

  Der Aufgabenbereich von Katharina Joksch war die Webentwicklung, genauer gesagt die Programmierung der digitalen Speisekarte.
  Neben der Planung der Datenbank und der sinnvollen Verwendung hilfreicher Frameworks entwickelte sie außerdem eine Java Applikation,
  die die Kommunikation zwischen der Speisekarte und der Drohne regelte.

  \subsection*{Alexander Punz}
  \textbf{Hardware \& Mechanik}

  Alexander Punz war sowohl für die Hardware als auch für die Mechanik verantwortlich. Seine Aufgaben waren sowohl die Konzeption
  und Produktion des Rotorschutzes, der einen sicheren Flug der Drohne ermöglichte, als auch die Herstellung diverser Halterungen,
  die für Sensorik, Transport und Flugtests verausgesetzt waren.

%%%%%%%%%%%%%%%%%%%%%%%%%%%%%%%%%%%%%%%%%%%%%%%%%%%%%%%%%%%%%%%%%%%%%%%%%%%%%%%
\section{Betreuer}
  \subsection*{Mag. Andreas Fink}
  Mag. Andreas Fink stand dem Projektteam als Hauptbetreuer der Abteilung für Informationstechnologie zur Seite.
  Seine objektive Sichtweise auf das Projekt, hat dem Team sehr geholfen den Fokus auf die Ziele zu legen und
  das Projekt in die Richtung zu entwickeln.
  Zusätzlich dazu betreute er individuell Markus Kaiser bei den Aspekten Projektleitung und Marketing.

  \subsection*{DI Herbert Fleck}
  DI Herbert Fleck war Hauptbetreuer der Mechatronik Abteilung unseres Teams. Er koordinierte den Prozess der Diplomarbeit
  gemeinsam mit Mag. Andreas Fink. Das Teeam schätzte außerdem sehr das Konstruktive Feedback bei Präsentationen.
  Er betreute nebenbei Lucas Ullrich mit Fachwissen aus dem Bereich der Elektronik.

  \subsection*{DI August Hörandl}
  DI August Hörandl fungierte als Individualbetreuer von Christina Bornberg. Duch seine Fähigkeiten und
  Erfahungen als Programmierer sowohl mit der Sprache C, als auch Java war das Entwickeln der Firmware,
  aber auch der Java-Applikation für die WLAN-Kommunikation wesentlich einfacher. Außerdem war
  DI August Hörandl unser Ansprechpartner wenn es Unklarheiten bei \LaTeX gab.

  \subsection*{MMag. Florian Weiss}
  MMag. Florian Weiss betreute Katharina Joksch bei der Entwicklung der digitalen Speisekarte. Sein umfangreiches
  Know-How im Bereich der Webentwicklung, dem Umgang mit diversen Frameworks und Bibliotheken half Katharina
  dabei ein Grundgerüst für die Entwicklung aufzubauen.

  \subsection*{DI Franz Temper}
  DI Franz Temper unterstützte Alexander Punz bei der Enticklung der Konstuktionen. Durch Fachwissen mit
  der Software Creo und Unterstützung bei 3D-Druck war es möglich, dass Alexander alle seiner Ziele erfolgreich
  umsetzen konnte.

%%%%%%%%%%%%%%%%%%%%%%%%%%%%%%%%%%%%%%%%%%%%%%%%%%%%%%%%%%%%%%%%%%%%%%%%%%%%%%%
\section{Partner / Sponsoren}

\subsection*{GRZ IT Center GmbH}
{GRZ IT Center\cite{grz}} ist eines der größten Rechenzentren Österreichs. Mit dem Fokus auf Bankenservicierung arbeitet das Unternehmen
mit Partnern wie Raiffeisen zusammen. Auf eine Anfrage für ein Sponsoring der Diplomarbeit erhielten wir die überaschende
Antwort, dass GRZ das Projekt sehr interessant findet, und uns aus diesen Gründen als Hauptsponsor, mit einem beträchtlichen Betrag
unterstützen würde.
Es wurde ein Sponsoringvertrag, auf Basis einer Mustervorlage aufgesetzt, in welchen beidseitig Konditionen für die Partnerschaft
verfasst wurden, und von beiden Parteien unterzeichnet.

\subsection*{OFI}
{Das Österreichische Forschungsinstitut\cite{ofi}} ist Experte die Prüfung für Werkstoffanwerundung
und Bauwerkserneuerungen. Neben finanzieller Unterstützung erhielt das Projektteam
ein professionelles, projektbegleitendes Team-Coaching, inklusive Zertifikat.

\subsection*{EVOtech GmbH}
Dank der Firma {EVOtech\cite{evotech}} war es möglich additive Module für den Hexacopter selbst anzufertigen.
Das gesamten Material, welches für den 3D-Druck benötigte wurde, wurde gesponsert.

\subsection*{LivingPhotos}
LivingPhotos ermöglichte uns, unsere Internetauftritte durch professionelle Teamfotos und Portraits aufzuwerten.

%%%%%%%%%%%%%%%%%%%%%%%%%%%%%%%%%%%%%%%%%%%%%%%%%%%%%%%%%%%%%%%%%%%%%%%%%%%%%%%
\section{Danksagung}
In erster Linie möchten wir uns bei unseren beiden Hauptbetreuern Mag. Andreas Fink und DI Herbert Fleck für die
Betreuung des gesamten Teams bedanken. Sowohl fachbezogene Ratschläge, als auch konstruktives Feedback wurden sehr
geschätzt und haben viel zur erfolgreichen Umsetzung der Diplomarbeit beigetragen.

Ebenfalls bedanken wir uns bei unseren Individualbetreuern DI August Hörandl, MMag. Florian Weiss und
DI Franz Temper, die dem Projektteam als Ansprechpersonen für besondere Fragestellungen zur Seite gestanden
sind.

Nicht zuletzt möchten wir unseren Dank unseren Partnern und Sponsoren, ganz besonders der GRZ IT Center GmbH
ausdrücken, da wir die Diplomarbeit nicht ohne die Unterstützung in diesem Rahmen umsetzen hätten können.

\chapter{Projektmanagement}
\renewcommand{\kapitelautor}{Autor: Markus Kaiser}

%%%%%%%%%%%%%%%%%%%%%%%%%%%%%%%%%%%%%%%%%%%%%%%%%%%%%%%%%%%%%%%%%%%%%%%%%%%%%%%
\section{Ziele}
Im folgenden Abschnitt werden zusammengefasst die Muss-, optionalen und Nicht-Ziele angeführt.
Die Formulierungen sind aus dem Kontext, der Ziele aus dem Projektantrag entnommen, und für die
Zusammenfassung umformuliert worden.

  \subsection{Muss-Ziele}
  \textbf{Zusammenbau des Multicopters}

  Der ausgewählte Multicopter ist soweit zusammengebaut, dass er flugfähig ist.
  Zusätzliche Komponenten, wie Fernbedienung, Akkus und Ersatzteile liegen vor.

  \textbf{Konzeptstudie: Sensorik}

  Um ohne manuelle Einfüsse fliegen zu können und sein Ziel zu finden braucht der Multicopter eine Reihe
  von Sensoren. Diese dienen zur Positions-bzw. zur Objekterfassung. Das Konzept beinhaltet außerdem eine
  Maßnahme für die Kommunikation zwischen Multicopter und Sensoren.
  Der Multicopter ist mit den Sensoren verbunden und bekommt von ihnen die notwendigen Informationen.

  \textbf{Konzeptstudie: Navigation}

  Es ist ein Konzept entworfen, wie der Multicopter die richtige Flugroute zum gewünschten
  Zielort findet. Dazu zählt sowohl der Teil der Firmware, der für den automatisierten Flug programmiert ist,
  als auch die Positionierung anhand von Sensoren und Wegmarkierungen.

  \textbf{Konzeptstudie: Objekterkennung}

  Es liegt ein Konzept vor, wie der Multicopter anhand von Sensoren Hindernisse oder Zielobjekte
  erkennen und diesen ausweichen, oder sie ansteuern kann.

  \textbf{Konzeptstudie: Sicherheit}

  Für den Fall eines Systemausfalls liegt ein Konzept vor, wie der Multicopter sicher landen kann.
  Weiters besteht die Möglichkeit, in den manuellen Flugmodus umzuschalten.

  \textbf{Testen der Konzeptstudien}

  Um festzustellen ob sich die erarbeiteten Konzepte auch in der Praxis bewähren und somit einen Einsatz zu
  teuren Kameratrackingsystemen schaffen zu können, sind diese umgesetzt. Als Sensoren sind eine Kamera
  sowie Ultraschallsensoren verwendet. Um weitere Positionen zu markieren arbeitet der Multicopter mit
  Farbcodes und/oder Infrarotsendern.

  \textbf{Speisekarte}

  Es ist eine digitale Speisekarte für ein autonomes Kellnersystem entwickelt.
  Die Anwendung läuft auf einem Tablet und bietet Funktionen wie die Auflistung der Speisen,
  die Bestellung und die Verarbeitung der Bestellung.
  Außerdem gibt es eine Möglichkeit dem Multicopter Informationen zu schicken.

  \textbf{Blog}

  Die Diplomarbeitswebsite fungiert in erster Linie als selbst programmierter Blog, um Interessenten
  jederzeit die Möglichkeit zu bieten, sich über den Status des Projektes zu informieren. Jedes Teammitglied
  kann im Backend-Bereich individuelle Blog- oder Entwicklertagebucheinträge über ein Benutzerkonto verfassen.

  \subsection{Optionale Ziele}
  \textbf{Erweiterte Funktionalitäten der Speisekarte}

  Administratoren haben bestimmte Verwaltungsmöglichkeiten.

  \textbf{Multicopter Erweiterungen}

  Platinen für die Kommunikation zwischen Multicopter und Basis sind angefertigt.
  Sensoren sind mittels Halterungen am Multicopter befestigt.
  Es gibt eine Möglichkeit ein Objekt auf dem Multicopter zu platzieren und es sicher zu transportieren.

  \textbf{Firmware Erweiterungen}

  Der Multicopter empfängt Informationen drahtlos von einem Server,
  wertet diese aus und verarbeitet sie im Anschluss.
  Grundfunktionen wie Steigen, Rollen oder Nicken sind programmiert.

  \subsection{Nicht-Ziele}
  \textbf{Mehrere Multicopter}

  Das System ist fähig, mehrere Multicopter gleichzeitig zu steuern, ohne Abstürze
  zu verursachen oder Menschen zu gefärden.

  \textbf{Sicherheitsmaßnahmen}

  Das Projekt enthält keine Sicherheitsmaßnahmen, die Verletzungen von von Menschen
  verhindern und das Risiko eines Absturzes senken.

  \textbf{Erfolgreiche Konzeptstudie}

  Das Ziel der Diplomarbeit ist es, dass das konzeptionierte System
  funktionsfähig umgesetzt ist.

%%%%%%%%%%%%%%%%%%%%%%%%%%%%%%%%%%%%%%%%%%%%%%%%%%%%%%%%%%%%%%%%%%%%%%%%%%%%%%%
\section{Management-Methode}
Das nachfolgende Kapitel beschäftigt sich mit den wesentlichen Unterschieden
zweier Projektmanagement-Methoden und der Wahl der Methode für diese Diplomarbeit.

  \subsection{Agile Methoden}
  Das Kennzeichen agiler Entwicklungsmethoden wie beispielsweise Scrum oder Kanban, sind
  ein iterativer Arbeitsprozess, Flexibilität, gute und häufige Kommunikation und vorallem
  Selbstständigkeit. Es wird kurz geplant und fortlaufend angepasst. Da nur bedingt eine Reihenfolge
  von zu erledigenden Tasks besteht finden sich diese Methoden oft in der Softwareentwicklung vor.

  \subsection{Klassische Methoden}
  Klassischen Entwicklungsmethoden zeichnen sich durch eine phasenorientierte Arbeitsweise aus.
  Ganz besonders die Planungsphasen sind sehr umfangreich, um das Risiko später auftretender Probleme oder
  Änderungen zu minimieren. Alle Ziele sind klar definiert und Tasks bis in kleinste Details ausformuliert.

  \subsection{Die gewählte Methode}
  Für die Entwicklung von Hovering Steward, war eine Kombination beider Methoden notwendig. Da,
  das Projekt sich sowohl aus einigen Software-, als auch Hardwarekomponenten zusammensetzt, war das
  Ziel sowohl die Vorteile der Flexibilität von agilen Methoden, als auch die ausführliche Planung der klassischen
  Methoden auszunutzen.

  Es entstand eine Mischung aus der agilen Entwicklungsmethode Scrum und der klassischen Methode Wasserfall.
  Für die Entwicklung der Firmware des Multicopters, aber auch die der webtechnologischen Komponenten eignete
  sich die flexible Arbeitsweise von Scrum, da nach Bedarf die einzelnen Module der Software entwickelt, und Schritt
  für Schritt in das System integriert wurden.
  Die Fertigung der Halterungen am Multicopter bedarfen jedoch einer genauen Planung von Reihenfolge von Arbeitsschritten,
  welche mithilfe eines Projektstrukturplans visualisiert wurden.

%%%%%%%%%%%%%%%%%%%%%%%%%%%%%%%%%%%%%%%%%%%%%%%%%%%%%%%%%%%%%%%%%%%%%%%%%%%%%%%
\section{Teammanagement / Teambuilding}
Da sich das Team aus sowohl aus Schülern verschiedener Klassen, aber auch Fachrichtungen
zusammensetzte war Teambuilding ein primärer Teil des Managements. Das Projekt bestand außerdem
aus Komponenten, die sich stark voneinader unterschieden, jedoch gut zusammenarbeiten mussten.

  \subsection{KaTeCos}
  Eine Form von internen Meetings, die den Grundstein für die Teamdynamik setzten, waren unsere
  sogenannten "Kaffe Team Coachings", kurz KaTeCo. Eine Projektmanegerin unseres Partnerunternehmens ofi
  bot sich als Coach für das Team an. Alle drei Wochen erhielten wird bei diesen Coachings wertvolle Techniken und
  Strategien, sowohl für das Teambuilding, als auch die Umsetzung des Projekts.

  \subsection{Projektkultur}
  In der Planungsphase wurde gemeinsam eine Projektkultur erarbeitet, die folgende Regeln und Pflichten
  für jedes Teammitglied festlegte.
  \begin{itemize}
    \item Wenn eine Deadline voraussichtlich nicht erfüllt werden kann, wird dem gesamten Team so früh wie möglich Bescheid gegeben.
    \item Ehrlichkeit und Offenheit gegenüber jedem Teammitglied
    \item Entscheidungen werden nicht nur vom Projektleiter, sondern vom gesamten Team getroffen
    \item Jedes Team hat immer Zugriff auf projektinterne Dateien und Informationen
    \item Wichtige e-Mails sind zuerst vom gesamten Team abzusegnen, bevor sie verschickt werden.
    \item Probleme werden nicht als Hindernis, sondern als Chance gesehen
  \end{itemize}
