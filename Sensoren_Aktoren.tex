% !TEX root = diplomarbeit.tex
\chapter{Sensoren}
\renewcommand{\kapitelautor}{Autor: Lucas Ullrich}

%%%%%%%%%%%%%%%%%%%%%%%%%%%%%%%%%%%%%%%%%%%%%%%%%%%%%%%%%%%%%%%%%%%%%%%%%%%%%%%
\section{PIXY CMUcam5}\label{PIXY_Headline}
Bei der PIXY CMUcam5 handelt es sich um ein Open Source Kameramodul, welches über eine Objekterkennung verfügt.
Mit diesem ist es möglich sogenannte Colorcodes oder einfache Objekte zu erkennen.

\begin{figure}[H]
  \begin{centering}
    \includegraphics[width = 0.4\textwidth]{Bilder/Pixy_CMUcam5}
  \par\end{centering}
  \caption{PIXY CMUcam5}
  \label{PIXY}
\end{figure}

\begin{figure}[H]
  \begin{centering}
    \subfigure[Colorcode]{\includegraphics[width = 0.4\textwidth]{Bilder/Colorcode}}
    \subfigure[Objekt]{\includegraphics[width = 0.4\textwidth]{Bilder/Objekt}}
  \par\end{centering}
  \caption{Erkennbare Objekttypen}
  \label{PIXY_Objekte}
\end{figure}

  \subsection{Umsetzung}
  Durch die PIXY CMUcam5 lässt sich eine relative Positionsmessung vergleichsweise einfach verwirklichen.
  Werden ein oder mehrere Objekte erkannt wird eine bestimmte Nummer (abhängig von der Farbe) sowie die Position am Bild und die Objektgröße übermittelt.
  Die Kamera arbeitet dabei mit einer Bildwiederholrate von $\SI{50}{\hertz}$, es ist also alle $\SI{20}{\milli\second}$ eine Auswertung des aktuellen Bildmaterials möglich.

  Die Kamera wird auf dem Hexacopter befestigt, mehrere Farbcodes kennzeichnen den Weg zu einem Tisch.
  Um an dieser Stelle eine Navigation zu erreichen wird der Hexacopter so gesteuert, dass er, abhängig von der Route,
  immer einen bestimmten Farbcode betrachtet, ist er über diesem sucht er den nächsten.

    \subsubsection{SPI Schnittstelle}
    Als Schnittstelle für die Kommunikation mit der Kamera wird eine SPI-Schnittstelle verwendet.
    Die Kamera selbst unterstützt unter anderem die seriellen Schnittstellen UART, I2C und SPI.
    Außerdem werden noch ein analoger und digitaler Output unterstützt, diese sind jedoch vergleichsweise beschränkt,
    da keine näheren Informationen zu dem Objekt übermittelt werden können sondern nur die Position beziehungsweise ob überhaupt ein Objekt erkannt wurde.

    Die SPI-Schnittstelle ist bei der Kamera besonders ausfallsicher. Hier wird ein Synchronisationsbyte gefordert, wird dieses nicht erkannt,
    zum Beispiel aufgrund eines Fehlers in der Datenübertragung, schickt die Kamera keine Daten.

    \subsubsection*{Überprüfen der SPI-Schnittstelle}
    Um zu überprüfen ob die SPI-Schnittstelle auch korrekt arbeitet, wird bei der ersten Inbetriebnahme der Output überprüft.
    Hierzu wird der Zustand der 3 Leitungen mit einem Oszilloskop betrachtet.
    \begin{itemize}
      \item \textcolor{blue}{Taktleitung}
      \item \textcolor{red}{Dateneingang (PIC)}
      \item \textcolor{green}{Datenausgang (PIC)}
    \end{itemize}

    \begin{figure}[tbh]
      \begin{centering}
        \subfigure[Großer Zeitbereich]{\includegraphics[width = 0.49\textwidth]{Bilder/SPI_gross}}
        \subfigure[Kleiner Zeitbereich]{\includegraphics[width = 0.49\textwidth]{Bilder/SPI_klein}}
      \par\end{centering}
      \caption{Ausgang der SPI Schnittstelle}
      \label{SPI-Ausgang}
    \end{figure}

    Der Wert mit dem diese Überprüfung durchgeführt wird, sollte möglichst variabel sein, hier wird 0xAA (1010 1010) verwendet.
    Wird dieser Wert nicht variabel angenommen, kann es dazu kommen, dass fälschlicher Weise angenommen wird, dass die Übertragung korrekt ist. Dies kann durch externe Faktoren wie
    Pull-Up-Widerstände oder Default-Status einer Leitung geschehen, in beiden Fällen würde je nach Schnittstelle "1" oder "0" übertragen werden, dies wäre falsch.

    Der Dateneingang des PIC, respektive der Ausgang der Kamera, zeigt eine deutliche Störung durch die Taktleitung an (siehe rote Linie Abbildung\ref{SPI-Ausgang}).
    Die beeinträchtigt bisher jedoch nicht die Funktion und wird daher nicht näher untersucht. Anhand der Ätzvorlagen konnte jedoch festgestellt werden, dass eine
    parallele Verlegung nicht gegeben ist.

    \subsubsection{Erkennen und Auswerten eines Bildes}
    Die Kamera schickt der Reihe nach die einzelnen Daten eines Objekts. Darunter ist auch ein Startwort welches ein neues Bild markiert.
    Mit den diversen Informationen zum Objekt ergeben sich folgende Daten:
    \begin{itemize}
      \item Neues Bild 0xAA55
      \item Objekt 0xAA55 oder Farbcode 0xAA56
      \item Checksum
      \item Objektnummer
      \item X-Position
      \item Y-Position
      \item Breite
      \item Höhe
      \item Drehwinkel, nur bei Farbcodes
    \end{itemize}
    Dabei ist die Objektnummer von den im Objekt oder Farbcode vorkommenden Farben abhängig, zusätzlich ist zu beachten, dass sie oktal dargestellt wird.
    Ein übermittelter Wert von dezimal 10, also oktal 12, bedeutet, dass die Farben 1 und 2 erkannt wurden.

    Will man nun ein neues Bild finden, muss man so lange nach 0xAA55 suchen bis man diese Daten gesendet bekommt.
    Anschließend gilt es noch festzustellen ob man einen Farbcode oder ein Objekt betrachtet,
    es muss also direkt darauf 0xAA56 oder nochmals 0xAA55 erkannt werden. Ist dies nicht der Fall, wurde kein neues Bild erkannt und man betrachtet ein normales Objekt.

    Betrachtet man nun die bis zum Erkennen eines neuen Bildes gesendeten Daten als gegenstandslos, ergibt sich eine vergleichsweise einfache Schleife um ein Bild zu erkennen.

    \lstset{language = c}
    \begin{lstlisting}
while(frame == 0) {
  w = ExchangeSpiWord(PIXY_SYNC, DUMMY);
  if(lw == PIXY_FRAME_OBJ && w == PIXY_FRAME_OBJ) {
  /*Frame detected, normal Object*/
    frame = 1;
    obj_type = 0;
    a_color[c_obj].type = PIXY_FRAME_OBJ;
  } else if(lw == PIXY_FRAME_OBJ && w == PIXY_COLORCODE) {
  /*Frame detected, Colorcode-Object*/
    frame = 1;
    obj_type = 1;
    a_color[c_obj].type = PIXY_COLORCODE;
  } else if(w == 0 && lw == 0){
  /*No frame detected*/
    frame = 0;
  }
  lw = w;
  c++;
  if(c > 254) {
    return 0;	//****Error, end of function
  }
}
    \end{lstlisting}
    Um nicht ewig in dieser Schleife fest zu hängen, wenn kein Bild erkannt wird und einen Fehler auslösen zu können wird die gesamte Funktion der Bildauswertung
    nach 255 Versuchen verlassen.

    Die weiteren Werte eines Objekts werden der Reihe nach in einer \gls{Struktur} abgespeichert, hier ist nichts Besonderes mehr zu beachten.

  \subsection{Herausforderungen und Lösungen}
  Insbesondere das Beispielprogramm\cite{PIXY_Porting_Examplecode} für eine Bildauswertung stellte einige Herausforderungen dar. Hier sind unterschiedliche Codeschnipsel in einem
  großen Beispiel zusammengefasst. Die große Ähnlichkeit verschiedener Funktionen erschweren die Lesbarkeit des Codes erheblich. So steht eine Funktion für die Erkennung eines
  neuen Bildes, diese zählt aber nur Bilder pro Sekunde, eine andere ist dann für die gesamte Auswertung und sucht dafür erneut nach einem Bild.

  Diese Zusammenhänge konnten durch einige Überlegungen erkannt werden und darauffolgend die eigene Firmware geschrieben werden.

  Ein weiteres Problem stellt die Empfindlichkeit der Kamera bei wechselnden Lichtverhältnissen dar. Da diese über die Farbsignatur eines Objektes arbeitet, werden bereits
  sehr kleine Differenzen unterschieden um verschiedene Objekte erkennen zu können beziehungsweise keine Störungen durch den Hintergrund zu erhalten. Dies wird jedoch ebenso in der
  anderen Richtung zu einem Problem. Wechselt das Licht ein wenig kann ein Farbcode oft nicht mehr zuverlässig erkannt werden und die Kamera muss über einen Computer nachkalibriert
  werden. Eine mögliche Lösung ist das konstante Beleuchten der Farbcodes mit einer sehr bestimmten Leuchtstärke, hier kann es leicht zu Überbelichtungen kommen, sowohl bei
  dunklen Verhältnissen, als auch die Umrüstung auf einen \gls{Infrarot}-Kit. Bei einem Infrarot-Kit fällt jedoch die Individualität jedes einzelnen Markers weg.

%%%%%%%%%%%%%%%%%%%%%%%%%%%%%%%%%%%%%%%%%%%%%%%%%%%%%%%%%%%%%%%%%%%%%%%%%%%%%%%
\section{Ultraschallsensor HC-SR04}
Der Ultraschallsensor HC-SR04 ist ein für Arduino entwickeltes Modul um Abstände zu messen. Die Messung geschieht durch das Aussenden von Ultraschallimpulsen,
die Messgröße wird dabei als laufzeitabhängiger Impuls retourniert.

\begin{figure}[tbh]
  \begin{centering}
    \includegraphics[width = 0.5\textwidth]{Bilder/Ultraschallsensor}
  \par\end{centering}
  \caption{Ultraschallsensor HC-SR04}
  \label{Ultraschallsensor}
\end{figure}

  \subsection{Technische Planung}
  Für die Steuerung des Hexacopter ist es notwendig die aktuelle Flughöhe zu wissen.
  Es ist mit einer bekannten Objektgröße und den von der Kamera vorliegenden Daten zwar möglich die aktuelle Flughöhe rechnerisch zu bestimmen,
  jedoch gestaltet sich dies sehr rechenaufwändig beziehungsweise ungenau. Um die Höhe möglichst einfach messen zu können bietet sich daher eine vergleichsweise langsame Messung,
  wie jene mit einem Ultraschallsignal an.
  Bei einer Schallgeschwindigkeit von $\SI{343}{\meter\per\second}$ entstehen bei einer zu messenden Distanz von $\SI{2}{\meter}$,
  Laufzeiten des Ultraschallsignals von \ca $\SI{12}{\milli\second}$ (Distanz mal 2 da das Signal wieder zurückkehren muss).

  \subsection{Umsetzung}
  Abhängig von der mit dem Mikroprozessor ermittelten Laufzeit $time\_height$ lässt sich jederzeit die aktuelle Flughöhe bestimmen.
  \[
  s(time\_height) = \frac{v \cdot t}{2} = \frac{\SI{343}{\meter\per\second} \cdot time\_height}{2}
  \]
  Die Höhe wird dabei in der main-Routine durch den Aufruf folgender Funktionen regelmäßig bestimmt:
  \lstset{language = c}
  \begin{lstlisting}
void StartHeightMeasure(void) {
  TMR5L = 0;
  TMR5H = 0;
  Trigger = 0;  //Start measurement
}

void ReadHeight(void) {
  while(TMR5GIF == 0);  //Wait for completed measurement
  TMR5GIF = 0;
  time_height = 0;
  time_height = TMR5H;  //Store time
  time_height <<= 8;
  time_height |= TMR5L;
  TMR5L = 0;
  TMR5H = 0;
  a_frame[0].height = time_height;
  a_frame_dif[0].dif_height = a_frame[1].height - a_frame[0].height;
  Trigger = 1;
}
  \end{lstlisting}

  In der ersten Funktion wird die Messung gestartet. Dazu wird ein Trigger-Signal an den Ultraschallsensor gesendet, dieser reagiert auf eine fallende Flanke.
  Zuvor wird darauf geachtet, dass die Register in denen die Zeit zurückgegeben wird auch wirklich leer (= 0) sind.
  In der zweiten Funktion folgt das Auslesen der vorhandenen Daten. Dazu werden die zwei 8-Bit Register in der 16-Bit Variable $time\_height$ abgespeichert.
  Zusätzlich wird die gemessene Höhe für die Auswertung in der $a\_frame[0].height$ Variable abgespeichert und die Differenz zur vorhergehenden Messung ermittelt.

  Auf eine Berechnung der genauen Höhe in m wird zu Gunsten der Verarbeitungszeit verzichtet, auf die spätere Auswertung der Flugdaten hat dies keinen Einfluss.

  \subsection{Herausforderungen und Lösungen}
  Bei der Verwendung des Ultraschallsensors traten einige Herausforderungen auf. So war bei einem der Testflüge ersichtlich, dass Throttle von der Firmware auf der Hardware nicht mehr
  zurück genommen wird, in der Simulation hingegen schon. Der erste Verdacht fiel auf eine verfälschte Messung im Betrieb, deshalb wurde in das Programm ein Teil implementiert, der
  es ermöglicht die gemessenen Werte über WLAN auf einen Computer zu übertragen.
  \lstset{language = c}
  \begin{lstlisting}
#ifdef DEBUG
void SendDebugInfo(unsigned int debug_info) {
    for(unsigned char debug_info_counter = 0;
    /*Split value with modulo for conversion into ASCII*/
      debug_info_counter < 5; debug_info_counter++) {
        send_info[debug_info_counter] = debug_info % 10;
        debug_info = ( debug_info - send_info[debug_info_counter]) / 10;
    }
    for(signed char debug_send_counter = 4;
    /*Send each character seperately*/
      debug_send_counter >= 0; debug_send_counter--) {
        unsigned char send = send_info[debug_send_counter];
        UartSendAscii(send);
    }
    UartSend(';');  //Divider-symbol
}
#endif
  \end{lstlisting}

  Um dieses nach der Verwendung wieder einfach aus dem Code entfernen zu können wurde das Übermitteln dieser Information an die Definition der $DEBUG$-Variable geknüpft.

  In der Übertragung war es wichtig auf die Auswertung des Computers zu achten, deshalb mussten ASCII-Zeichen übertragen werden.

  Die erste Messung bestätigte die Erwartungen:
  \begin{figure}[H]
    \begin{centering}
      \includegraphics[width = 0.7\textwidth]{Bilder/Hoehenmessung}
    \par\end{centering}
    \caption{Höhenmessung während einem Flugvorgang}
    \label{Hoehenmessung}
  \end{figure}
  Die Messung wurde während einem manuellen Flug gestartet, die Gasstellung war auf mehr als Schwebeflug, und wurde mit einer Landung beendet. Der Hexacopter wurde dabei ständig
  auf der selben Höhe gehalten, die Messwerte änderten sich dennoch und zeigten eine Änderung der Flughöhe an.

  Als Grund wurden erst Störungen des Ultraschallsensors durch Geräusche des Hexacopters vermutet, dies konnte jedoch nicht bestätigt werden, hingegen dessen führten Vibrationen
  zu den falschen Messwerten. Um die Vibrationen nahezu vollständig zu beseitigen wurde der Sensor gedämpft gelagert und ebenso das Kabel anders verlegt.
  \begin{figure}[H]
    \begin{centering}
      \includegraphics[width = 0.7\textwidth]{Bilder/Hoehenmessung_gedaempft}
    \par\end{centering}
    \caption{Gedämpfte Höhenmessung während einem Flugvorgang}
    \label{Hoehenmessung_gedaempft}
  \end{figure}
  Das Ergebnis zeigt, während einem Flug, deutlich stabilere Werte, eine Annäherung zum Boden wird ebenso sehr zuverlässig erkannt. Lediglich bei der Messung der Distanz in einer Flughöhe
  von \ca $\SI{1.3}{\meter}$ (Ausgangsposition für die Messungen) zeigen sich starke Schwankungen.

\chapter{Aktoren}
\renewcommand{\kapitelautor}{Autor: Christina Bornberg}

%%%%%%%%%%%%%%%%%%%%%%%%%%%%%%%%%%%%%%%%%%%%%%%%%%%%%%%%%%%%%%%%%%%%%%%%%%%%%%%
\section{Multicopter}
  Multicopter gehören zu der Luftfahrzeuggattung Hubschrauber. Sie starten senkrecht und können durch den Antrieb der Rotoren Neigung erzeugen. Durch diese Neigung kann der Hexacopter nach vorne, zurück, nach links und nach rechts fliegen. Durch die Rotoren, bei denen sich abwechselnd einer nach links und einer nach rechts dreht, kann sich der Hexacopter zusätzlich um seine Hochachse drehen.

  \subsection{Funktion}
  Ein Multicopter besteht im Normalfall aus folgenden Komponenten:
    \begin{itemize}
   \item Das \textbf{Empfängermodul} des Multicopters kann von einem Sender Steuersignale empfangen und aufgrund der empfangenen Steuersignale Soll-Werte für die Motoren berechnen, um den Multicopter auf die entsprechende Geschwindigkeit in x, y oder z Richtung zu bringen.
   \item Der \textbf{Flightcontroller} ist der zentrale Regler und steuert die einzelnen Rotoren an.
   \item Die \textbf{Multicopterhardware} besteht aus Rotoren, Motoren, einem Gestell und der Leistungselektronik. 
   \end{itemize}


    \begin{figure}[H]
    \begin{centering}
      \includegraphics[width = \textwidth]{Bilder/bor_copter_funk}
    \par\end{centering}
    \caption{Aufbau eines einfachen Multicopters}
    \label{Funktion_Multicopter}
  \end{figure}

   \subsection{Erweiterte Funktion}
    Im Rahmen dieser Diplomarbeit wurden die üblicherweise verwendeten Teile um vier weitere Komponenten erweitert, um einen automatischen Flug zu gewährleisten: eine Pixy CUMcam5, ein Ultraschallsensor, ein WLAN Modul und ein Mikrocontroller. Pixycam und Ultraschallsensor sind dafür verantwortlich, die Daten für die Berechnung der augenblicklichen Position des Hexacopters zu bestimmen. Im Mikrocontroller läuft ein Algorithmus, der aus diesen empfangen Daten die Position überprüft. Wenn der Wert vom optimalen Bereich abweicht, werden die Steuersignale abgeändert. Durch das Wlan-Modul bekommt der Mikrocontroller die Information, durch wie viele Marker der Weg definiert ist und welcher Farbcode der letzte ist, auf dem die Landung erfolgt.


  \subsection{Multicopter Arten}
  Es gibt verschiedene Arten von Multicoptern\cite{GrundlagenMulticopter}.

  Je nachdem wie viele Rotoren in einer Ebene liegen, setzt sich der Name zusammen.
  Die bekannteste Art ist der Quadrocopter, welcher vier Rotoren besitzt. Weiters gibt es Hexacopter mit sechs und Octocopter mit acht Rotoren.

  Der Flightcontroller DJI Naza-M lite unterstützt verschiedene Konfigurationsarten\cite{NAZA_Konfig} von Quadrocoptern und Hexacoptern. Hierzu zählen + und x Konfiguration für den Quadrocopter. Beim Hexacopter gibt es die Varianten +, x, reverse Y und Y. Diese werden im nächsten Abschnitt genauer beschrieben. 

  \subsection{Konfiguration}
  Grundsätzlich gibt es 2 Arten einen Multicopter zu Konfigurieren \cite{GrundlagenMulticopter}. 
  Einerseits die +- beziehungsweise I-Konfiguration, andererseits die x- beziehungsweise H-Konfiguration. Weiters gibt es die weniger verbreitete Y-Konfiguration. Dabei geht es um die Ausrichtung und die damit verbundene Fluglage.

  Für diese Diplomarbeit wurde die + Konfiguration für Hexacopter gewählt.

  Wie auf folgender Grafik\cite{NAZA_Konfig} ersichtlich ist, gibt es nach links und nach rechts drehende Rotoren. Würden sich alle in die selbe Richtung drehen, würde sich der Multicopter um die eigene Hochachse drehen. Die rot gekennzeichneten Arme geben an, wo die Vorderseite des Multicopters ist. Bei der Y-Konfiguration sind die blau markierten Propeller oben, die roten unten. 

    \begin{figure}[H]
      \begin{centering}
        \includegraphics[width = \textwidth]{Bilder/bor_copter_konfig}
      \par\end{centering}
      \caption{Konfigurationsarten von Multicoptern}
      \label{Flussdiragramm}
    \end{figure}

  \subsection{Mögliche Anwendungsgebiete}
  Multicopter finden in vielen Bereichen Anwendung\cite{copterAnwendung}. 
  \begin{itemize}
    \item Fotographie und Videos / Filmindustrie / Anfertigen von hoch aufgelösten Luftbildkarten
    \item Multicopter als Hobby / Kunstflug
    \item Lagerhallen und Logistik / Bestands- und Inventaraufnahmen im Straßenbau
    \item Forschung / Schwarmverhalten
    \item Multicopter als Lebensretter / Rettungseinsätze / Katastrophenschutz
    \item Unterstützung der Polizei / große Menschenmengen überwachen
    \item Unbemannter Aufklärer bei Spezialeinheiten
    \item Kontrolle, Inspektion und Dokumentation von Brücken, Gebäuden, Gräben
    \item Militärische Anwendung: Spionagedrohne zur Aufklärung, Kampfdrohne zur Zerstörung, Rettungungsdrohne für Hilfsaktionen
  \end{itemize}


\section{Rotoren}

  \subsection{Technische Planung}
  Autor: Lucas Ullrich\\

  Der Hexacopter selbst verfügt über 6 Rotoren. Diese werden für die unterschiedlichen Flugrichtungen von dem Flightcontroller angesteuert.
  Der Flightcontroller liest die Daten an den einzelnen Steuerpins als Servo-Impulse ein. Die Daten werden dabei in einer Periode von $\SI{20}{\milli\second}$ gesendet.
  Die Information selbst liegt in den ersten $\SI{2}{\milli\second}$. Für den Impuls beträgt die Mittelstellung $\SI{1.5}{\milli\second}$, das Minimum $\SI{1}{\milli\second}$
  und das Maximum $\SI{2}{\milli\second}$. Das Signal folgt somit den Konventionen eines Servo-Impulses.

    \subsubsection{Steuerungsarten}
    Autor: Christina Bornberg\\

    Der Flightcontroller DJI NAZA-M lite wird über die Befehle\cite{GrundlagenMulticopter} Aileron, Elevator, Rudder und Throttle gesteuert. Zusätzlich gibt es den 2/3-position mode channel, der für das Umschalten mehrerer Modi zuständig ist. Handelsübliche Fernbedienungen verfügen über die selben Steuerbefehle. 

      \begin{itemize}
        \item \textbf{A: Aileron} auch Rollen (engl. roll) genannt, ist für die Bewegung nach Links und Rechts zuständig.
        Um diese auszuführen, werden bei der Beschleunigung nach links, die rechten Propeller stärker betrieben, umgekehrt drehen sich die auf der linken Seite befindlichen Propeller beim Flug nach rechts schneller. Durch die entstehende Neigung, fliegt der Hexacopter in die gewünschte Richtung.
        \item \textbf{E: Elevator} auch Nicken (engl. pitch) genannt, ist für die Vorwärts- und Rückwärtsbewegung zuständig.
        Beim nach vorne und nach hinten fliegen, werden ebenfalls die Rotoren schneller betrieben, die auf der jeweils gegenüberliegenden Seite liegen.
        \item \textbf{R: Rudder} auch Gieren (engl. yaw) genannt, ist für die Rotation an der Hochachse zuständig.
        Um den Hexacopter um seine eigene Hochachse rotieren zu lassen, werden für eine Rotation nach rechts, die rechts-drehenden Rotoren schneller betrieben, bei der Rotation nach Links, die links-drehenden.
        \item \textbf{T: Throttle} reguliert die Höhe.
        Bei gleichmäßiger Ansteuerung der sechs Rotoren, kann man je nach Drehgeschwindigkeit die Höhe verändern. Der Hexacopter fliegt bei stärkerer Beschleunigung nach oben, ansonsten sinkt er. Dies wird auch als Uplift and Downfall bezeichnet.
        \item \textbf{U: 2/3-position switch channel}\cite{positionswitch}ist zum Umschalten von zwei oder drei Steuermodi zuständig. Im Fall unserer Diplomarbeit wird zwischen autonomem und manuellem Flug gewechselt. 
      \end{itemize}

    \begin{figure}[H]
      \begin{centering}
        \includegraphics[width = \textwidth]{Bilder/bor_fernbedienung}
      \par\end{centering}
      \caption{Fernsteuerung mit den Steuerbefehlen}
      \label{Fernsteuerung}
    \end{figure}

  \subsection{Umsetzung}
  Um den Hexacopter steuern zu können müssen die einzelnen Impulse vom Mikrocontroller imitiert werden.
